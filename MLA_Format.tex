\documentclass[12pt]{article}

%Margins
\usepackage[letterpaper]{geometry}
\geometry{top=1.0in, bottom=1.0in, left=1.0in, right=1.0in}

%Font
\usepackage{times}

%Easier citing
\usepackage[backend=biber, style=mla]{biblatex}s
%change name of cited file below
\addbibresource{•.bib}

%Doublespacing
\usepackage{setspace}
\doublespacing

%rotating tables sideways if it's too wide for the paper, you never know when it's useful
\usepackage{rotating}

%Adding apostrophes is going to be a bitch
%Add apostrophes by typing "\textquotesingle"
\usepackage{textcomp}

%header/page numbering, very important to MLA format
\usepackage{fancyhdr}
\pagestyle{fancy}
\lhead{}
\chead{}
%Make sure to replace "Last" with last name in the next line
\rhead{• \thepage}
\lfoot{}
\cfoot{}
\rfoot{}
\renewcommand{\headrulewidth}{0pt}
\renewcommand{\footrulewidth}{0pt}

%To make sure that we actually have a header 0.5in away from the top edge
%12pt is one-sixth of an inch. Subtract this from 0.5in to get headsep value
\setlength{\headsep}{0.333in}

%Works Cited
%%to start, use \begin{workscited}, each entry starts with \bibent
%%%NOTE: I'm able to change "\bibent" to something different like a variable
\newcommand{\bibent}{\noindent \hangindent 40pt}

\begin{document}
	\begin{flushleft}
		%First page name, class, etc
		•\\ %Name
		•\\ %Professor's name
		•\\ %Class
		•\\ %Date, syntax day-monthName-year

		%Title
		\begin{center}
			%Paper title here
				•
		\end{center}

		%Changes paragraph indentation to 0.5in
		%You can reuse this command to remove indents
		\setlength{\parindent}{0.5in}
		%Begin body of paper here
		•
		\newpage
		\begin{center}
		%Works cited goes here, make sure to use \bibent for every entry
		\printbibliography
		\end{center}
	\end{flushleft}
\end{document}
